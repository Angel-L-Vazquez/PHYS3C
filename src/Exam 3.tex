%! Author = angel

% Preamble
\documentclass[11pt]{article}

% Packages
\usepackage{amsmath,amssymb,amsthm}
\usepackage[letterpaper,margin=0.75in]{geometry}
\usepackage{pgfplots}
\usepackage{gensymb}
\usepackage{tikz}
\usepackage[usenames, dvipsnames]{xcolor}
\usepackage{fancyhdr}
\pgfplotsset{compat=1.18}
\usepackage{tikz-cd}

\usetikzlibrary{decorations.markings, arrows.meta}
% Document
\tikzset{
    marrow/.style={decoration={markings,mark=at position 0.5 with {\arrow{#1}}}, postaction=decorate}
}


\newcommand{\fahrenheit}{\degree \text{F}}

% Document
\begin{document}
    \noindent \textbf{Exam 3 Cheat Sheet}% Document
    \\ \noindent \newline
    Speed of light in a vacuum:
      \begin{equation}
        c  =  3.0 \times 10^8 \text{ m/s} \tag{in a vacuum}
    \end{equation}
    Speed of light where $n$ is the index of refraction:
    \begin{equation}
      v = \frac{c}{n} \tag{speed of light}
    \end{equation}
    Law of refraction:
\begin{equation}
  n_1 \sin \theta_1 = n_2 \sin \theta_2 \tag{Snell's Law}
\end{equation}
When light refracts along the boundary ($n_2 > n_1$, light traveling from less to more dense):
\begin{equation}
  \theta_c = \sin (\frac{n_2}{n_1}) \tag{Critical angle}
\end{equation}
Beam of light going through polarizing sheets:
\begin{equation}
  I_1 = \frac{1}{2} I_0 \tag{one half rule}
 \end{equation}
   Going through another sheet:
   \begin{equation}
     I = I_1 \cos^2 (\Delta \theta) \tag{cosine-squared}
    \end{equation}

\end{document}

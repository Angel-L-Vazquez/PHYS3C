%! Author = angel

% Preamble
\documentclass[11pt]{article}

% Packages
\usepackage{amsmath,amssymb,amsthm}
\usepackage[letterpaper,margin=0.75in]{geometry}
\usepackage{pgfplots}
\usepackage{gensymb}
\usepackage{tikz}
\usepackage[usenames, dvipsnames]{xcolor}
\usepackage{fancyhdr}
\pgfplotsset{compat=1.18}
\usepackage{tikz-cd}

\usetikzlibrary{decorations.markings, arrows.meta}
% Document
\tikzset{
    marrow/.style={decoration={markings,mark=at position 0.5 with {\arrow{#1}}}, postaction=decorate}
}


\newcommand{\fahrenheit}{\degree \text{F}}

% Document
\begin{document}
    \noindent \textbf{Chapter 20: Entropy }% Document
    \\ \noindent \newline We have focused with reversible processes so far,
    however most processes in the universe are irreversible.
    Imagine microwaving a bag of popcorn, as the popcorn heats up,
    the kernels pop, but you can never unpop those kernels.
    We see that when an irreversible process occurs,
    the closed system sees an increase in \textbf{entropy.}
    Like the unpopped kernel, entropy never decreases, because in doing so,
    it would make the irreversible process reversible.
    We define entropy S ($J/K$) as:
    \begin{equation}
        \Delta S = \int_{i}^f \frac{1}{T} dQ \tag{entropy}
    \end{equation}
    However, there is an issue with this definition.
    In a free expansion process, there is no heat transfer,
    and thus the system does not pass through equilibrium states,
    causing different parts of the gas to fill in unevenly.
    In other words, $dQ$ is not clearly defined, making integration impossible.
    Since entropy is a state function,
    we determine its change by considering a
    \textbf{reversible isothermal process}
    that connects the same initial and final states:
    \begin{equation}
        \Delta S = \frac{Q}{T} \tag{entropy, isothermal process}
    \end{equation}
    We consider entropy a state function, since $\Delta S$ between different states only
    depend on the initial and final conditions.
    For an ideal gas, entropy change can be rewritten using the ideal gas law as:
    \begin{equation}
        \Delta S = nR\ln (\frac{V_f}{v_i}) + nC_v \ln (\frac{T_f}{T_i}) \tag{all procresses}
    \end{equation}
    To find the total change in entropy of a closed system,
    we need to consider the change in heat transfer for both the object, and environment.
    Imagine an ice cube placed in a bucket of warm water.
    We would find the $\Delta S$ for the water reaching its melting point,
    then for the heat energy used for the latent heat of fusion, and then for the water
    to reach the temperature of the surrounding water.
    Now repeat the process, considering the environment by finding $Q$ for the three parts (note that $T$ is constant).
    Then finally subtracting the two: $\Delta S_{ice} - \Delta S_{env}$

    \noindent \\Another example is an engine.
    Focusing particularly on one ideal type, the Carnot Engine.
    During each cycle, the working substance (such as steam or gas)
    absorbs energy $|Q_H|$ as heat from a thermal reservoir at a constant
    temperature $T_H$ and discharges energy $|Q_L|$ as heat to a second
    thermal reservoir at a lower constant temperature $T_L$.
    We find work with:
    \begin{equation}
        W = |Q_H| - |Q_L| \tag{Carnot engine work}
    \end{equation}
    In a Carnot Engine, there are two reversible energy transfers as heat.
    The net entropy change per cycle:
    \begin{equation}
        \Delta S = \Delta S_H + \Delta S_L = \frac{|Q_H|}{T_H} - \frac{|Q_L|}{T_L} \tag{entropy, carnot engine}
    \end{equation}
    $\Delta S_H$ is positive as energy $|Q_H|$ is added to the working substance as heat,
    while $\Delta S_L$ is negative because energy $|Q_L|$ is removed.
    We measure the efficiency:
    \begin{equation}
        \epsilon = 1 - \frac{T_C}{T_H} = 1 - \frac{|Q_L|}{|Q_H|} \tag{carnot efficiency}
    \end{equation}
    We generalize this for any engine as:
    \begin{equation}
        \epsilon = \frac{\text{energy we get}}{\text{energy we used}} = \frac{|W|}{|Q_H|} \tag{efficiency, all engines}
    \end{equation}



\end{document}
%! Author = angel

% Preamble
\documentclass[11pt]{article}

% Packages
\usepackage{amsmath,amssymb,amsthm}
\usepackage[letterpaper,margin=0.75in]{geometry}
\usepackage{pgfplots}
\usepackage{gensymb}
\usepackage{tikz}
\usepackage[usenames, dvipsnames]{xcolor}
\usepackage{fancyhdr}
\pgfplotsset{compat=1.18}
\usepackage{tikz-cd}
\usepackage{booktabs}

\usetikzlibrary{decorations.markings, arrows.meta}
% Document
\tikzset{
    marrow/.style={decoration={markings,mark=at position 0.5 with {\arrow{#1}}}, postaction=decorate}
}


\newcommand{\fahrenheit}{\degree \text{F}}

% Document
\begin{document}
    \noindent \textbf{Chapter 19: Kinetic Theory of Gases}
    \\ \noindent \newline Gases consist of atoms that take up space, applying pressure within a volume.
    Since these atoms move around, we can assign them a temperature as well.
    To work on this atomic level, it is best to work in terms of moles,
    where we can use \textbf{Avogadro's number} (mol$^{-1}$):
    \begin{equation}
        N_A = 6.02 \times 10^{23} \text{ mol}^{-1} \tag{Avogadro's number}
        \end{equation}
    \noindent To find the amount of \textbf{moles} $n$, we would need to find the ratio
    between the amount of \textbf{molecules} $N$ and Avogadro's number:
    \begin{equation}
       n = \frac{N}{N_A} \tag{moles}
    \end{equation}
    \noindent We can also use the mass of the sample M$_{\text{sam}}$ (g)
    with the molar mass M (mol, mass of a single mole) or molecular mass $m$ (g, the mass of a single molecule):
    \begin{equation}
        n = \frac{M_{\text{sam}}}{M} = \frac{M_{\text{sam}}}{mN_A} \tag{moles}
    \end{equation}
    \noindent Take special note, a mol is \textbf{not} the same as a molecule.
    A mol measures a specific amount, usually atoms or molecules, in the same as a dozen measures a quantity,
    while a molecule refers to a group of atoms chemically bonded such as H$_2$O.


    \noindent \\Through experimentation, scientists found that at low enough densities,
    all gases follow this relation:
    \begin{equation}
        pV = nRT \tag{Ideal Gas Law}
    \end{equation}
    where $p$ is absolute pressure Pa, $V$ is volume, $n$ is the number of moles,
    $R$ is the gas constant, and $T$ is the temperature in Kelvin.
    \begin{equation}
        R = 8.31\text{ } J / (mol \cdot K) \tag{gas constant}
    \end{equation}
    We can rewrite the gas law formula to get it in terms of molecules $n$:
    \begin{equation}
        pV = NkT \tag{Ideal Gas Law}
    \end{equation}
    Where k is the Boltzmann constant:
    \begin{equation}
         k = \frac{R}{N_A} = 1.38 \times 10^{-23} J/K \tag{Boltzmann constant}
    \end{equation}
    Now lets take a look when the temperature is constant,
    which is an \textbf{isothermal process.}
    When finding work, we find:
        \begin{equation}
            W = nRT\ln\frac{V_f}{v_i} \tag{ideal gas, isothermal process}
        \end{equation}
    Holding volume gives us:
    \begin{equation}
        W = 0 \tag{constant-volume process}
    \end{equation}
    Holding pressure yields:
    \begin{equation}
        W = p \Delta V \tag{constant-pressure process}
    \end{equation}

    If we take a closer look into the properties of gas,
    we see that the molecules move in all directions with differing speeds,
    bumping into each other, and into the walls of its confined space.
    To find the average speed of a given set of molecules:
    \begin{equation}
        v_{avg} = \frac{\sum v_i}{N} \tag{average}
    \end{equation}
    Since these molecules are moving at wildly different speeds,
    the average velocity is not enough, so finding the root-square means
    will give us a more meaningful average speed:
    \begin{equation}
        v_{\text{rms}} = \sqrt{\frac{\sum V_i^2}{N}} \tag{v_{\text{rms}}}
    \end{equation}
    As each molecule carries energy moving around in a confined space,
    at any instant its translational kinetic energy is $\frac{1}{2}mv^2$
    when it bumps into another molecule.
    Its average translational kinetic energy over time is:
    \begin{equation}
        K_{avg} = \frac{1}{2} mv_{rms}^2 \tag{average kinetic energy}
    \end{equation}
    Substituting out $v_{rms}$ we find:
    \begin{equation}
        K_{avg} = \frac{3}{2} kT \tag{average kinetic energy}
    \end{equation}
    Continuing the motion of molecules,
    the movement of any given molecule appears chaotic,
    changing paths and speeds as it collides with other particles.
    However, the distance between collisions is linear at a constant velocity
    known as a mean free path:
    \begin{equation}
        \lambda = \frac{1}{\sqrt{2} \pi d^2 N/V} \tag{mean free path}
    \end{equation}
    Where $d$ is the diameter of a molecule,
    and $N/V$ is the number of molecules per per unit volume (density of molecules).
    So far we have a general idea of how the molecular speeds,
    however we can find the distribution, using Maxwell’s speed distribution law:
    \begin{equation}
        P(v) = 4 \pi (\frac{M}{2 \pi RT})^{3/2} \cdot v^2 e^{-Mv^2/2RT} \tag{distribution}
    \end{equation}
    Graphing this gives us a curve, where we can use the total area under the curve
    as a fraction for the molecular speeds:
    \begin{equation}
        \int_{0}^{\infty} P(v) \text{ } dv = 1
    \end{equation}
    Now we can solve for each of the speeds, Where temperature is in kelvin,
    and $M$ is molar mass ($kg/mol$):
    \begin{equation}
        v_{avg} = \int_{0}^{\infty} v P(v) \text{ } dv = \sqrt{\frac{8RT}{\pi M}}
    \end{equation}
    \begin{equation}
    (v)^2_{avg} = \int_{0}^{\infty} v^2 P(v) \text{ } dv = \frac{3RT}{ M} \Rightarrow v_{rms} = \sqrt{\frac{3RT}{M}}
    \end{equation}
    The most probable speed gives us the speed where $P(v)$ is at its max.
    \begin{equation}
        v_p = \sqrt{\frac{2RT}{M}}
    \end{equation}
    \newpage
    Whe find that gasses can either be monatomic, diatomic or polyatomic, which relates to
    how many atoms a given gas molecule has.
    Knowing heat is related to temperature change:
    \begin{equation}
        Q = nC \Delta T 
    \end{equation}
    Where we can use the molar specific heat at a constant pressure $C_p$ or
    at a constant volume $C_v$:
    \begin{equation}
        C_v = \frac{f}{2} R
    \end{equation}
    $f$ refers to the degrees of freedom a gas has.
    For monatomic gasses $f = 3$, and for diatomic gasses $f =5$.
    With this we can find the change in internal energy for any ideal gas.
    \begin{equation}
        \Delta E_{int} = nC_{\text{v}} \Delta T \tag{ideal gas, all processes}
    \end{equation}
    Rewriting the first law of thermodynamics nets us:
    \begin{equation}
        W = p \Delta V = nR \Delta T
    \end{equation}
    So we can solve for $C_p$:
    \begin{equation}
        C_p = C_v + R
    \end{equation}
    In an Adiabatic expansion of a gas, $Q = 0$.
    This gives the relation:
    \begin{equation}
        pV^{\gamma} = \text{a constant} \tag{adiabatic process}
    \end{equation}
    Where $\gamma = C_p/C_v$, which another relation is formed:
    \begin{equation}
        p_i V_i^{\gamma} = p_f V_f^{\gamma} \tag {adiabatic process}
    \end{equation}
    Free expansion, no change in temperature, also nets a similar result:
    \begin{equation}
        p_i V_i = p_f V_f \tag {free expansion}
    \end{equation}

    \noindent Now we can summarize each of the 4 special cases:
    \begin{table}[h]
        \centering
\renewcommand{\arraystretch}{1.5}
        \begin{tabular}{lp{3cm}lp{6cm}l}
            \toprule
            \textbf{Process} & \textbf{Constant} & \textbf{Special Results} \\
            \midrule
            All & &  $\Delta E_{int} = Q - W$ and $\Delta E_{int} = n C_v \Delta T$\\
            Isobaric  & p  & $Q = nC_p \Delta T$, $W = p \Delta V$\\
            Isothermal & T  & $Q = W = nRT \ln (\frac{v_f}{v_i}) $, $\Delta E_{int} = 0$ \\
            Adiabatic & $pV^{\gamma}$,$TV^{\gamma - 1}$ & $Q = 0$, $W = - \Delta E_{int} $ \\
            Isochoric & V  & $Q = \Delta E_{int} = n C_v \Delta T$, $W = 0$ \\
            \bottomrule
        \end{tabular}
    \end{table}
\end{document}